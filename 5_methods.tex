% !TEX root = 0_main.tex

\section{Methods}\label{sec:methods}

This study employs a multi-modal qualitative approach that combines data collected from 25 online and face-to-face (F2F) semi-structured interviews, ten months of participant observation, focus groups, and documentary analysis of 55 documents, mainly internal AOD documents provided to linguists and official blog posts from \textit{\url{blog.amara.org}}. Table \ref{tab:participants} and Table \ref{tab:participants2} provides an overview of the main characteristics of the participants with whom we conducted semi-structured interviews and organised focus groups. The table includes their gender, main role\footnote{This refers to the main tasks carried out by the participant in AOD. For example, as discussed in Section \ref{sec:case-study} ,whether they are part of the core team. The term DQA refers to Designated Quality Assurer. DQAs are responsible for managing large, active projects where clients often request special instructions. Further details are discussed in Section \ref{subsec:models}.} in AOD, number of years in AOD, location and language groups they belong to (only for linguists), among others. 

%PARTICIPANTS TABLE
%\begin{longtable}{p{.10\textwidth}p{.05\textwidth}p{.15\textwidth}p{.20\textwidth}p{.10\textwidth}p{.05\textwidth}p{.25\textwidth}} 
\begin{table}
\begin{tabular}{lllllll}
\hline
\begin{tabular}[c]{@{}l@{}}Participant\\ID\end{tabular} & Gender & \begin{tabular}[c]{@{}l@{}}Main role\\in AOD\end{tabular} & \begin{tabular}[c]{@{}l@{}}Main translation \\ languages\end{tabular} & \begin{tabular}[c]{@{}l@{}}Country of \\ residence\end{tabular} & \begin{tabular}[c]{@{}l@{}}Years \\ in AOD\end{tabular} & \begin{tabular}[c]{@{}l@{}}Related research \\ method\end{tabular} \\ \hline
%\endhead
P\textsubscript{1} & Male & \begin{tabular}[c]{@{}l@{}}Linguist \end{tabular} & German & Germany & 2 & \begin{tabular}[c]{@{}l@{}}Online semi-structured \\ interview\end{tabular} \\
P\textsubscript{2} & Male & Linguist and DQA & \begin{tabular}[c]{@{}l@{}}Marathi and Hindi \end{tabular} & India & 3 & \begin{tabular}[c]{@{}l@{}}Online semi-structured \\ interview\end{tabular} \\
P\textsubscript{3} & Female & Linguist & Greek & \begin{tabular}[c]{@{}l@{}}United States\\of America\end{tabular} & 7 & \begin{tabular}[c]{@{}l@{}}Online semi-structured \\ interview\end{tabular} \\
P\textsubscript{4} & Female & Linguist & \begin{tabular}[c]{@{}l@{}}Russian and English\end{tabular} & \begin{tabular}[c]{@{}l@{}}United States\\of America\end{tabular} & 3 & \begin{tabular}[c]{@{}l@{}}Online semi-structured \\ interview\end{tabular} \\
P\textsubscript{5} & Male & Linguist and DQA & Greek & Greece & 4 & \begin{tabular}[c]{@{}l@{}}Online semi-structured \\ interview\end{tabular} \\
P\textsubscript{6} & Male & Linguist & Arabic & Egypt & 2 & \begin{tabular}[c]{@{}l@{}}Online semi-structured \\ interview\end{tabular} \\
P\textsubscript{7} & Male & Linguist & \begin{tabular}[c]{@{}l@{}}Dutch and Spanish\end{tabular} & Chile & 6 & \begin{tabular}[c]{@{}l@{}}Online semi-structured \\ interview\end{tabular} \\
P\textsubscript{8} & Female & Linguist and DQA & \begin{tabular}[c]{@{}l@{}}Hindi and English\end{tabular} & India & 2 & \begin{tabular}[c]{@{}l@{}}Online semi-structured \\ interview\end{tabular} \\
P\textsubscript{9} & Female & Linguist & \begin{tabular}[c]{@{}l@{}}Simplified Chinese, \\ Traditional Chinese \\ and Malay\end{tabular} & Malaysia & 2 & \begin{tabular}[c]{@{}l@{}}Online semi-structured \\ interview\end{tabular} \\
P\textsubscript{10} & Female & Linguist and DQA & English & \begin{tabular}[c]{@{}l@{}}United States\\of America\end{tabular} & 5 & \begin{tabular}[c]{@{}l@{}}Online semi-structured \\ interview\end{tabular} \\
P\textsubscript{11} & Female & Linguist & \begin{tabular}[c]{@{}l@{}}French and\\Romanian\end{tabular} & Brazil & 1 & \begin{tabular}[c]{@{}l@{}}Online semi-structured \\ interview\end{tabular} \\
P\textsubscript{12} & Female & Linguist & Greek & Netherlands & 3 & \begin{tabular}[c]{@{}l@{}}Online semi-structured \\ interview\end{tabular} \\
P\textsubscript{13} & Female & Linguist & Russian & Russia & 2 & \begin{tabular}[c]{@{}l@{}}Online semi-structured \\ interview\end{tabular} \\
P\textsubscript{14} & Male & \begin{tabular}[c]{@{}l@{}}Linguist \end{tabular} & \begin{tabular}[c]{@{}l@{}}Portuguese-Portugal\end{tabular} & Portugal & 6 & \begin{tabular}[c]{@{}l@{}}Online semi-structured \\ interview\end{tabular} \\
P\textsubscript{15} & Male & Linguist & Polish & Poland & 2 & \begin{tabular}[c]{@{}l@{}}Online semi-structured \\ interview\end{tabular} \\
P\textsubscript{16} & Male & Linguist & Swedish & Sweden & 4 & \begin{tabular}[c]{@{}l@{}}Online semi-structured \\ interview\end{tabular} \\
P\textsubscript{17} & Female & Accountant (core) & N/A & \begin{tabular}[c]{@{}l@{}}United States\\of America\end{tabular} & 5 & \begin{tabular}[c]{@{}l@{}}F2F semi-structured \\ interview\end{tabular} \\
P\textsubscript{18} & Female & Project Manager (core) & N/A & \begin{tabular}[c]{@{}l@{}}United States\\of America\end{tabular} & 7 & \begin{tabular}[c]{@{}l@{}}F2F semi-structured \\ interview\end{tabular} \\
P\textsubscript{19} & Male & Project Leader (core) & N/A & \begin{tabular}[c]{@{}l@{}}United States\\of America\end{tabular} & 8 & \begin{tabular}[c]{@{}l@{}}F2F semi-structured \\ interview\end{tabular} \\
P\textsubscript{20} & Female & Project Leader (core) & N/A & \begin{tabular}[c]{@{}l@{}}United States\\of America\end{tabular} & 3 & \begin{tabular}[c]{@{}l@{}}F2F semi-structured \\ interview\end{tabular} \\ 
\multicolumn{7}{c}{} \\
\multicolumn{7}{r}{\textbf{Continued on Table \ref{tab:participants2}}} \\ \hline
\end{tabular}
\caption{The participants' main characteristics.}
\label{tab:participants}
\end{table}

\begin{table}
\begin{tabular}{lllllll}
\hline
\begin{tabular}[c]{@{}l@{}}Participant\\ID\end{tabular} & Gender & \begin{tabular}[c]{@{}l@{}}Main role\\in AOD\end{tabular} & \begin{tabular}[c]{@{}l@{}}Main translation \\ languages\end{tabular} & \begin{tabular}[c]{@{}l@{}}Country of \\ residence\end{tabular} & \begin{tabular}[c]{@{}l@{}}Years \\ in AOD\end{tabular} & \begin{tabular}[c]{@{}l@{}}Related research \\ method\end{tabular} \\ \hline
\multicolumn{7}{r}{\textbf{Continued from Table \ref{tab:participants}}} \\
\multicolumn{7}{c}{} \\
P\textsubscript{21} & Non-binary & Developer (core) & N/A & \begin{tabular}[c]{@{}l@{}}United States\\of America\end{tabular} & 2 & \begin{tabular}[c]{@{}l@{}}F2F semi-structured \\ interview\end{tabular} \\ 
P\textsubscript{22} & Female & Project Manager (core) & N/A & Brazil & 6 & \begin{tabular}[c]{@{}l@{}}Online semi-structured \\ interview\end{tabular} \\
P\textsubscript{23} & Non-binary & Recruiter (core) & N/A & \begin{tabular}[c]{@{}l@{}}United States\\of America\end{tabular} & 4 & \begin{tabular}[c]{@{}l@{}}Online semi-structured \\ interview\end{tabular} \\
P\textsubscript{24} & Female & \begin{tabular}[c]{@{}l@{}}Customer service\\(core)\end{tabular} & N/A & Spain & 6 & \begin{tabular}[c]{@{}l@{}}Online semi-structured \\ interview\end{tabular} \\
P\textsubscript{25} & Female & Linguist and DQA & \begin{tabular}[c]{@{}l@{}} Portuguese-Brazilian, \\ English and Spanish\end{tabular} & Spain & 5 & \begin{tabular}[c]{@{}l@{}}F2F semi-structured \\ interview and focus \\ group\end{tabular} \\ 
P\textsubscript{26} & Female & Linguist and DQA & Portuguese-Brazilian & Brazil & 4 & Focus group \\
P\textsubscript{27} & Male & Linguist & Portuguese-Brazilian & Brazil & 7 & Focus group \\
P\textsubscript{28} & Male & Linguist and DQA & Portuguese-Brazilian & Brazil & 5 & Focus group \\
P\textsubscript{29} & Female & Linguist and DQA & Portuguese-Brazilian & Brazil & 3 & Focus group \\
P\textsubscript{30} & Female & Linguist & Portuguese-Brazilian & Brazil & 5 & Focus group \\ \hline
\end{tabular}
\caption{The participants' main characteristics (continuation).}
\label{tab:participants2}
\end{table}
%\end{longtable}

The collected data were coded following an ethnographic content analysis approach \cite{altheide1987reflections}, which involved a continuous process of discovery and comparison of key categories emerging from the data. The various analytical tasks were supported by the Computer‐Assisted Qualitative Data Analysis Software NVivo 12. %Despite the continuous nature, inherent to the taken ethnographic approach, there were two distinctive phases regarding data collection and analysis. 

%TO-KEEP: OLD METHODS TABLE. To keep just in case.

%\begin{table*}[ht]
%\caption{Overview of data collection methods by phases.}
%\label{tab:methods}
%\begin{tabular}{lll}
%\hline
%Method & Phase 1 & Phase 2 \\
%\hline
%Participant observation & \begin{tabular}[c]{@{}l@{}}Field notes created during offline and online \\ participant observation from October 2018 to \\ March 2019\end{tabular} & \begin{tabular}[c]{@{}l@{}}Field notes created during offline and online \\ participant observation from March 2019 to \\ July 2020\end{tabular} \\
%Semi-structured interviews & \begin{tabular}[c]{@{}l@{}}15 semi-structured interviews with linguists from several \\ language groups: Traditional Chinese, Arabic, Greek, \\ Swedish, Portuguese-Brazil, etc.\end{tabular} & \begin{tabular}[c]{@{}l@{}}9 semi-structured interviews with members of \\ the community with a wide range of roles: project \\ managers, developers, co-founders, etc.\end{tabular} \\
%Documentary analysis & 33 internal and public documents & 22 blog posts, mainly from \url{blog.amara.org} \\
%Focus groups & N/A & \begin{tabular}[c]{@{}l@{}}Two-day workshop with several focus group\\sessions with six linguists of the\\ Portuguese-Brazilian team.\end{tabular}\\
%\hline
%\end{tabular}
%\end{table*}

\subsection{Participant observation and interviews} %: October 2018 - March 2019}

%This phase focused on an in-depth analysis of the inner workings of AOD from the perspective of the linguists. 
Online participant observation was carried out over six months (October 2018 - March 2019) to engage with the day-to-day practices of AOD linguists: from the recruitment and onboarding processes to the execution of regular tasks, such as captioning. In addition, 17 semi-structured interviews (see P\textsubscript{1} - P\textsubscript{16} in Table \ref{tab:participants} and P\textsubscript{25} in Table \ref{tab:participants2}) were conducted following a purposive sampling \cite{palys2008purposive} intended to gather the diversity of linguists in terms of language group, experience level, and degree of engagement. The data collected provided us with a rich picture of the experiences, needs and vision of the workflow of an %in participating as an 
AOD linguist. The primary outcomes of this part of the research were the mapping of the workflow of AOD and the identification of an initial set of communitarian needs which led us to discover several intervention points as potential areas to experiment with the development of worker-centric tools to support crowdsourced labour.

A similar approach was conducted but this time with core members of AOD. It involved four months of online participant observation (April 2019 - July 2019), eight semi-structured interviews (see P\textsubscript{17} - P\textsubscript{24} in Table \ref{tab:participants} and Table \ref{tab:participants2}) and documentary analysis of materials generated and posted in the official channels of AOD. As well as with the linguists, the semi-structured interviews were conducted following a purposive sampling \cite{palys2008purposive} with key members of AOD's core team considering the diverse roles in AOD, i.e., project managers, developers, members of the finance team, and project leaders, among others. The data analysis carried out here allowed us to further our understanding of the organisational processes of the workflow and the changes experienced in it over time. The aim was to include all of the different perspectives of the actors involved in the platform, to supplement the information gathered from the linguists. More importantly, the analysis of these data led us to select our point of intervention: task allocation. Task allocation is a necessary precursor to working. As a result, task allocation represents a suitable starting point for envisioning more cooperative labour processes.

\subsection{Focus Groups}

Interviews and participant observations were followed up with an online two-day workshop that included several focus group sessions (organised in June 2020). A call for participation was disseminated through the official AOD channels, including a short survey to show interest in involvement. From all of the linguistic groups in AOD, we chose the Portuguese-Brazilian due to its high degree of organisational complexity. We selected six linguists (see P\textsubscript{25} - P\textsubscript{30} in Table \ref{tab:participants2}) according to their different degrees of experience, since we aimed to have a variety of backgrounds. These focus group sessions allowed us, together with the linguistics, to identify alternative models for allocating tasks. The identified models were subsequently validated by the AOD's core team.

\subsection{Ethical considerations}

The ethical principles described by the European Research Council \cite{erc_ethics} were followed, as well as the recommendations from the Association of Internet Researchers \cite{markham2012}. Drawing on these guidelines, we constantly reassessed so that the discovery of any new issues resulted in remedial action. These actions include anonymising participants and references to customers in field notes and transcripts, in addition to the use of information sheets and consent forms to participate in the interviews and the focus groups.