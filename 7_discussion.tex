% !TEX root = 0_main.tex

\section{Discussion}
\label{sec:discussion}

Conceptualising more egalitarian working conditions in crowdsourcing in collaboration with workers is not unique to this study \cite{kittur2013future,graham2018towards}. Nevertheless, our focus on alternate forms of task allocation \textit{within} a cooperative crowdsourcing organisation — as opposed to platforms like AMT — is. Most large crowdsourcing platforms are less cooperative than AOD, making it challenging to alter aspects of the platform. This study illustrates the potential of co-designing hand-in-hand with workers and the significance of task allocation in dictating the nature of work. % Nevertheless, there remains a lack of research exploring how to engage workers in re-imagining a more transparent and equal process for obtaining work.% Projects such as The Drivers Cooperative, a worker-run ride-hailing cooperative in New York City, are heading in this direction.

Previous research has criticised algorithmic task allocation — typical in ridesharing platforms like Uber and Lyft — for its opacity and control over workers' profiles, routes, customers, and wages. As reported by Gray and Suri in \cite{gray2019ghost}, and confirmed in this research, current task allocation methods used in crowdsourcing platforms, which are primarily governed by an FCFS logic, generate competitive dynamics that result in an unfair and unequal distribution of labour and a sense of frustration among workers. We also found that the inherent competitiveness imposed by FCFS also harms the relationship between workers, negatively impacting the sense of community.

Our research in collaboration with AOD's workers allowed us to define alternative methods to FCFS for task allocation. Through a multi-modal methodology that included interviews, participant observation, focus groups and documentary analysis, we identified together with AOD's workers three models that could allocate tasks considering their needs.

The review of the literature demonstrates that most task allocation approaches are focused on optimising the needs of task requesters. These task-requester-centric solutions are intended to reduce costs \cite{ho2012online}, maximise matching \cite{machado2016task}, and increase the quality of results \cite{yu2013bringing}. Fairness in task allocation has been discussed by Fu and Liu \cite{fu2021fairness}. However, their solution focuses on creating fair crowdsourcing workflows, i.e., a logically related series of tasks, by minimising costs. In contrast, our work aims to shift the current FCFS logic of task allocation for models designed by workers to create cooperative and more equitable working conditions.

In this sense, round-robin (RR), one of the identified models, was proposed by the participants to create a more balanced workload. In RR, tasks are pre-assigned to workers in rounds, reducing the competitive dynamics of the current FCFS allocation practice. Although there was consensus among  participants that having a model that pre-assigns tasks in rounds could improve their experience in the platform, there are still essential implementation details in question. In the forthcoming focus groups with linguists, we will examine the rules and parameters that will define the operation of the model. For example, how does the availability of workers impact the model? How does the complexity of tasks affect the assignation of labour? How does worker expertise influence the distribution of tasks?

In discussing workers' expertise, the participants expressed that a reputation-based model might be a valuable complement to RR. In this context, workers' reputation would be captured through a system that reflects workers' performance based on historical feedback. Although performance-based reputation systems are already part of crowdsourcing platforms (e.g., Waze, TaskRabbit, AMT), participants see this model as applicable to improve transparency in the promotion mechanisms used in AOD. Here, workers need to build a ``career path'' by moving through different stages to reach higher levels of responsibility within the organisation. The significant concern with the reputation model, according to participants, is discrimination against newcomers and low-rated workers as well as task concentration by highly-rated workers. Brawley and Pruy have also mentioned the harmful impacts of reputation systems in crowdsourcing platforms in \cite{brawley2016work}. The participants suggest distributing more straightforward tasks to novices to promote inclusiveness, ensuring that they have assigned tasks when joining the platform.
As with the case of RR, there are implementation details to be addressed, and we plan to organise specific focus groups with linguists to explore them. It is still unclear, for example, what aspects of workers' history should be factored into one's reputation. Apart from workers' level and their historical performance, some participants suggested other elements such as attention to past deadlines, caring work \cite{10.3389/fhumd.2021.618207} (e.g., welcoming and tutoring newcomers), and consideration of colleagues' feedback. Gamification techniques based on artefacts that represent participants' reputation (e.g., badges, points, ranking) \cite{feng2018gamification} were also discussed as potential features to equip reputation-based models. 

A task allocation model that includes features of the tasks (e.g., complexity, size, topic, length) and workers' skills, background, and preferences was also seen by the participants as a suitable complement to reputation systems. Apart from complementing the reputation model, this method has been found advantageous to match workers with their expertise and skills resulting in high-quality tasks. Alternatively, some participants indicated that assigning tasks that are not necessarily related to workers' abilities and background might favour learning opportunities for workers. The parametrisation of this content-based model was also left for the next round of focus groups with the participants. It remains unclear how workers' domain of expertise should be included in the model and how to operationalise workers' preferences and interests.

The results presented in this article cannot, however, be generalised and should be understood within the particular case of AOD and similar crowdsourcing platforms. Moreover, due to our qualitative approach, the results cannot be generalised within AOD, considering other groups of linguists may relay a significantly different context, and therefore other models might be more suitable for them. In order to tackle this, we are currently carrying out a longitudinal quantitative analysis of the relationship between users and their activities, drawing on the data already available on AOD’s platform. Drawing on this data, we plan to carry out a similar research process with linguists of other language groups (e.g., English-Japanese, English-Italian, English-Spanish).

Furthermore, we want to explore how to develop tools which allow crowdsourcing workers to decide on the models to use in different contexts. Similarly, these tools to support decision-making could be employed to determine collectively how a model could be parameterised. In doing so, we plan to  develop collaborative decision-making tools that leverage the affordances \cite{doi:10.1177/21582440211002526, 10.3389/fbloc.2021.577680} of distributed-ledger technologies, such as blockchains, to allow crowdsourcing workers to prioritise parameters within the models which better suit their needs, as well as to decide collectively between the models themselves. In the context of our case study, a blockchain-based solution might enable linguists belonging to different language groups to self-organise in Decentralised Autonomous Organizations (DAOs) and resolve task allocation mechanisms that satisfy their requirements\footnote{See \cite{el2020overview} for an overview of how organisations have been using blockchain technologies with the aim to decentralise governance.}. Given that the results are part of a work-in-progress research endeavour, whether the identified models help to reduce competitiveness and improve  working conditions is to be validated.
