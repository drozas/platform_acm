% !TEX root = 0_main.tex

\section{Conclusion}\label{sec:conclusion}
This article reports on an endeavour to engage crowdsourcing workers in a multi-modal user-centred research process to identify alternative models of value distribution in crowdsourcing platforms. Three models have emerged as a result of the process, namely (i) round-robin, (ii) content-based, and (iii) reputation-based. Although the proposed models have the potential to improve the workers' experience in crowdsourcing platforms by distributing tasks more fairly, implementation details need to be discussed in subsequent research. Our aim is for this line of research to impact beyond this case study to broader areas of the platform economy. Ultimately, our goal is to foster workers' participation in and ownership of the platforms that mediate their work. Similar platforms owned by cooperatives of workers, distributing tasks and value according to the agreements reached by them, could be envisioned for a variety of areas. An example could be a cooperative of taxi drivers whose organisation is mediated by a platform which they control. The platform could distribute and monitor rides and payments according to the rules defined by the workers within their specific context.

 In sum, despite the inherent limitations due to the ongoing nature of this research, the models identified throughout the collaboration with AOD show us that it is possible to envision more cooperative models to distribute work, in which the producers progressively become owners of the means of production and the fruits of their labour expressions of self-realisation \cite{hansson_capitalizing_2018} and, as a result, \textit{platforms might belong to those who work on them}.